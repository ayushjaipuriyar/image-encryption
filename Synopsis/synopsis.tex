% Preamble
% ---
\documentclass[11pt,a4paper,english]{article}

% Packages
% ---
\usepackage{amsmath} % Advanced math typesetting
\usepackage[utf8]{inputenc} % Unicode support (Umlauts etc.)
\usepackage[english]{babel} % Change hyphenation rules
\usepackage{hyperref} % Add a link to your document
\usepackage{graphicx} % Add pictures to your document
\usepackage{listings} % Source code formatting and highlighting
\usepackage{makeidx}
\bibliographystyle{plain}

% ---
\begin{document}

\title{Synopsis}
\begin{titlepage}
    \begin{center}
        \LARGE
        \textbf{A Minor Project Synopsis}\\
        on\\
        \Huge
        \textbf{Chaos Based Image Encryption}\\
        \vspace{0.5cm}
        \Large
        Submitted to Manipal University Jaipur\\
        towards the partial fulfillment for the award of the degree of\\
        \vspace{0.5cm}
        \LARGE
        \textbf{Bachelor of Technology}\\
        \textbf{In Information Technology}\\
        \vspace{0.5cm}
        \large
        By\\
        \Large
        \textbf{Ayush Jaipuriyar}\\
        209302167\\
        \textbf{Pratham Dhiman}\\
        209302249\\
        Section E\\
        \vspace{0.5cm}
        \large
        Under the Guidance of\\
        \Large
        \textbf{Dr. Anju Yadav }

        \vfill

        \includegraphics[width=11.58cm,height=2.28cm]{logo}

        \LARGE
        \textbf{Department of Information Technology}\\
        \textbf{School of Information Technology }\\
        \textbf{Manipal University Jaipur}\\
        \Large
        \textbf{2022-2023}

    \end{center}
\end{titlepage}
\newpage
\date{}
\title{}
\maketitle
\tableofcontents
\newpage
\section{Introduction}
Encryption of images is important in today's digital world as it helps to protect sensitive and confidential information contained in images from unauthorized access. With the increasing use of digital images for personal, business and government purposes, there is a growing need to ensure that this data is protected from hackers, cyber-criminals and other malicious actors who might use the information for their own purposes. Encryption provides a secure and effective means of protecting images by transforming them into an unreadable form, making it difficult for unauthorized individuals to access or view the original image. This helps to ensure the privacy and confidentiality of the information contained in the image and reduces the risk of data breaches and theft.
\subsection{Encryption}
Encryption is the process of converting plaintext into an unreadable form (ciphertext) to protect its confidentiality and prevent unauthorized access to its content. It uses mathematical algorithms (cipher) and a key to transform the data. Only those with the corresponding decryption key can reverse the process and access the original plaintext. Encryption helps to ensure the privacy and security of data transmitted over networks or stored on devices.
\subsection{Chaos Theory}
Chaos theory is a branch of mathematics that studies the behavior of dynamic systems that are highly sensitive to initial conditions, leading to seemingly random patterns and unpredictable behavior. It deals with how small differences in initial conditions can lead to vastly different outcomes over time, which is often referred to as the butterfly effect. Chaos theory has applications in various fields, such as physics, economics, and weather forecasting.

Chaos theory provides several important features to image encryption, including:
\begin{enumerate}
    \item \textbf{Sensitivity to Initial Conditions}: The sensitivity to initial conditions in chaotic systems provides a high degree of security as small differences in the initial conditions can lead to vastly different results, making it difficult for an attacker to determine the encryption process.
    \item \textbf{Unpredictability:} The unpredictable behavior of chaotic systems adds an extra layer of security as the encrypted image appears to be random noise, making it difficult for an attacker to determine the original image.
    \item \textbf{High-dimensional Dynamics:} The high-dimensional dynamics of chaotic systems allow for the creation of complex encryption algorithms that can provide a high degree of security.
    \item \textbf{Ease of Implementation:} The mathematical simplicity of some chaotic systems makes them easy to implement in software, providing a convenient means of encrypting images.
\end{enumerate}
Overall, chaos theory provides a means of creating secure, efficient, and easy-to-implement image encryption algorithms that can help protect sensitive information transmitted over communication networks.
\newpage
\section{Motivation}
Our motivation for developing a new chaos-based image encryption technique is a result of our passion for cryptography and our fascination with chaos theory. I believe that this research will not only advance the field of cryptography but also help to improve the security of sensitive information. The challenge of developing a new image encryption technique appeals to me and I am eager to explore the unique properties of chaos-based encryption and apply them to image encryption. This work presents a significant challenge and an opportunity for me to push the boundaries of what is currently possible. Successfully developing a new chaos-based image encryption technique would be a major accomplishment for me, both professionally and personally, and would be a source of pride and satisfaction in our career as a researcher.
\paragraph{}
\newpage
\section{Project Objectives}
\paragraph{}
The objectives of the project are out lined as follows :-
\subsection{Confidentiality}Confidentiality in image encryption refers to the protection of sensitive information in an image file by encoding it in a way that can only be deciphered by authorized parties. The goal is to prevent unauthorized access to the original information and maintain the privacy of the data. This is achieved through various encryption techniques that scramble the original image data and transform it into a secure format that can only be decrypted using a specific key or passwor
\subsection{Robustness}Robustness in image encryption refers to the ability of an encryption algorithm to withstand various types of attacks, such as statistical attacks, differential attacks, and brute force attacks, while maintaining the confidentiality and security of the encrypted image. A robust encryption algorithm should be able to resist these attacks, even if an attacker has partial knowledge of the encryption process or the encrypted image. The goal of robustness in image encryption is to ensure that the encrypted image cannot be easily decrypted, even if an attacker has access to the encrypted data and some knowledge of the encryption process.
\subsection{Effieciency}Efficiency in image encryption refers to the speed and resource utilization of an encryption algorithm. The efficiency of an encryption algorithm is an important factor to consider when choosing an encryption method for images, as images tend to be large in size and require a significant amount of processing power to encrypt and decrypt. A highly efficient encryption algorithm will have a low computational overhead and use minimal memory resources, allowing for fast encryption and decryption times. Additionally, the encryption algorithm should not significantly degrade the quality of the image, as this could make the encrypted image less usable. A balanced approach is necessary when considering efficiency in image encryption, as a highly efficient algorithm that is easily broken is not secure, and a highly secure algorithm that is inefficient is not practical for real-world applications.\\
\subsection{Key management}Key management in image encryption refers to the processes and procedures used to generate, store, distribute, and secure the encryption keys used to encrypt and decrypt images. Effective key management is crucial to the security of encrypted images, as an attacker who gains access to the encryption key can easily decrypt the image. To ensure the security of encrypted images, key management should follow industry best practices, such as using strong encryption algorithms, employing secure key storage mechanisms, implementing proper key distribution procedures, and regularly updating and rotating encryption keys. Additionally, key management should be scalable, allowing for the easy distribution and management of encryption keys for a large number of users or devices. Effective key management is essential for ensuring the confidentiality and security of encrypted images and is an important aspect of any image encryption system.
\subsection{Adaptability}Adaptability in image encryption refers to the ability of an encryption algorithm to adapt to changing security needs and technological advancements. As technology and security threats evolve, encryption algorithms should be updated to ensure that they remain secure. An adaptable encryption algorithm should also be able to handle different types of images, such as grayscale, color, and high-resolution images, and should be able to adjust to changes in image size, format, and data structure. Additionally, an adaptable encryption algorithm should be able to integrate with existing security infrastructure and should provide mechanisms for upgrading encryption keys and adding new security features. Adaptability is important in image encryption as it helps ensure that the encryption algorithm remains secure and effective over time, protecting encrypted images from evolving security threats.
\subsection{Compatibility}Compatibility in image encryption refers to the ability of an encryption algorithm to work with different software platforms, hardware devices, and communication protocols. A compatible encryption algorithm should be able to encrypt and decrypt images on a variety of devices and operating systems, allowing encrypted images to be securely transmitted and stored. Additionally, a compatible encryption algorithm should be able to integrate with other security systems, such as firewalls, intrusion detection systems, and access control systems, to provide a comprehensive security solution. Compatibility is important in image encryption as it enables the encrypted images to be securely transmitted and stored across a variety of platforms, devices, and networks, and helps ensure that encrypted images remain confidential and secure.
\subsection{Transparency}Transparency in image encryption refers to the visibility of the encryption process and the encrypted image. A transparent encryption algorithm should encrypt images in such a way that they can be easily decrypted and viewed without noticeable degradation in quality. This is important for applications where the encrypted image needs to be processed or analyzed, as the encryption process should not interfere with the image data. Additionally, a transparent encryption algorithm should not add significant overhead to the encryption and decryption process, and should not affect the performance of other systems that process the encrypted image. Transparency is important in image encryption as it enables encrypted images to be used in real-world applications without affecting their functionality or quality.
\newpage
\section{Methodology}
\newpage
\section{Faculties reqired for proposed work}
\newpage
\section{Bibliography}
% \print
\bibliography{Synopsis}
\nocite{*}
\end{document}